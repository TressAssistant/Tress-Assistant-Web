\documentclass[12pt]{article}

\usepackage{amsmath, amssymb}
\usepackage{geometry}
\geometry{margin=1in}
\usepackage{hyperref}
\usepackage{enumitem}
\usepackage{bm}
\usepackage{physics}
\usepackage{graphicx}
\usepackage{color}
\usepackage{mathtools}
\setlength{\parskip}{0.8em}

\begin{document}

\title{Formal Explanation of Hair Color Blending Formulas (Revised for Realism)}
\author{}
\date{}
\maketitle

\section*{Introduction}
This document presents a mathematically rigorous model for simulating the blending of natural and artificial (dyed) hair colors. The model reflects the physical limitations of hair dyeing: lightness (brightness) of natural hair strongly limits the achievable lightness after dyeing, while hue and saturation are more blendable, especially for light hair or yellow-based dyes. The formulation is suitable for computer graphics implementations (e.g., OpenGL, GLSL).

\section*{Variables and Ranges}
Let:
\begin{itemize}
    \item \textbf{Natural hair color in HSL:} \\ $N = (N_h, N_s, N_l)$
    \begin{itemize}
        \item $N_h$: Hue in radians, $[0, 2\pi)$
        \item $N_s$: Saturation, $[0, 1]$
        \item $N_l$: Lightness, $[0, 1]$
    \end{itemize}
    \item \textbf{Natural hair color in RGB:} \\ $(N_r, N_g, N_b)$ from HSL$\to$RGB, each $[0, 1]$
    \item \textbf{Natural brightness:} \\ $NB = 0.299 N_r + 0.587 N_g + 0.114 N_b$
    \item \textbf{Artificial color in HSL:} \\ $A = (A_h, A_s, A_l)$, same ranges as above
    \item \textbf{Final blended color in HSL:} \\ $F = (F_h, F_s, F_l)$
\end{itemize}

\section*{Blending Equations}
If the hair is bleached (i.e., \texttt{bleached = true}), the final blended color is simply the artificial color:
\(
F = A
\)

Otherwise, the final blended color is a weighted average of natural and artificial HSL components, but with \textbf{distinct blending factors for hue, saturation, and lightness}:
\begin{align*}
F_h &= \alpha \cdot A_h + (1 - \alpha) \cdot N_h \\
F_s &= \beta \cdot A_s + (1 - \beta) \cdot N_s \\
F_l &= \gamma \cdot A_l + (1 - \gamma) \cdot N_l
\end{align*}
Where
\begin{itemize}
    \item \texttt{bleached}: boolean, true if hair is bleached
    \item $\alpha$: blending factor for hue
    \item $\beta$: blending factor for saturation
    \item $\gamma$: blending factor for lightness (special constraints, see below)
\end{itemize}

% Special handling for achromatic colors
\begin{itemize}
    \item \textbf{Special handling for achromatic colors (white, black, gray):}
    \begin{itemize}
        \item If both natural and artificial saturation are zero (i.e., both are achromatic), set the final hue to any value (e.g., 0 or the natural hue), as it will not affect the RGB result. Do not perform hue blending in this case to avoid NaN.
        \item If only one color is achromatic, use the hue of the chromatic color for blending.
    \end{itemize}
\end{itemize}

\begin{align*}
F_h &= \begin{cases}
  \text{any value (e.g., } N_h \text{)}, & \text{if } N_s = 0 \text{ and } A_s = 0 \\
  A_h, & \text{if } A_s > 0 \text{ and } N_s = 0 \\
  N_h, & \text{if } N_s > 0 \text{ and } A_s = 0 \\
  \alpha \cdot A_h + (1 - \alpha) \cdot N_h, & \text{otherwise}
\end{cases} \\
F_s &= \beta \cdot A_s + (1 - \beta) \cdot N_s \\
F_l &= \gamma \cdot A_l + (1 - \gamma) \cdot N_l \\
\\
\alpha &= S(NB) + (1 - S(NB)) \left( \frac{1 + \cos(A_h - \frac{\pi}{3})}{2} \right)^p \\
\gamma &= S_{\text{dark}}(A_l - N_l) + (1 - S_{\text{dark}}(A_l - N_l)) NB^q
\end{align*}
where
\(
S(NB) = \frac{1}{1 + e^{-k(NB - c)}}
\)
\(
S_{\text{dark}}(x) = \frac{1}{1 + e^{d x}} \quad (d \gg 1)
\)

\section*{Parameter Recommendations}
\begin{itemize}
    \item $p = 2$: sharp proximity for yellow range
    \item $k = 5$, $c = 0.5$: sigmoid transition at mid-brightness
    \item $q = 1$ (linear) or $2$ (quadratic, more realistic for lightness anchoring)
    \item $d = 20$: steepness for darkening transition
    \item All HSL values normalized to $[0,1]$ except hue in $[0, 2\pi)$
\end{itemize}

\section*{Practical Effects}
\begin{itemize}
    \item \textbf{Dark hair}:
    \begin{itemize}
        \item Final lightness stays low, regardless of dye lightness.
        \item Only yellow-based dyes shift hue/saturation noticeably.
    \end{itemize}
    \item \textbf{Light hair}:
    \begin{itemize}
        \item Artificial color can smoothly darken hair (e.g., blonde to black becomes black), with no abrupt transition.
        \item Artificial color cannot lighten hair beyond its natural brightness.
    \end{itemize}
    \item \textbf{Intermediate hair}:
    \begin{itemize}
        \item Gradual, nonlinear, and smooth transition in dominance.
    \end{itemize}
\end{itemize}

\section*{Implementation Notes}
\begin{itemize}
    \item Always handle hue blending circularly (e.g., via trigonometric interpolation or modular arithmetic).
    \item \textbf{For achromatic colors (saturation = 0), skip hue blending and set hue to any value (e.g., 0 or the natural hue) to avoid NaN.}
    \item For graphics code, ensure all variables are clamped to valid ranges after computation.
\end{itemize}

\section*{Summary}
This revised model ensures realistic simulation:
\begin{itemize}
    \item \textbf{Hue and saturation} can shift with dye, especially for compatible colors and light hair.
    \item \textbf{Lightness} is strongly anchored to the natural base, reflecting real-world dye limitations.
    \item \textbf{Nonlinear transitions} ensure that extreme results (like light yellow on black hair) are impossible.
\end{itemize}
This model is robust for digital hair color preview and simulation applications.

\end{document}
